\documentclass{report}
\usepackage[T1]{fontenc} % Fontes T1
\usepackage[utf8]{inputenc} % Input UTF8
\usepackage[backend=biber, style=ieee]{biblatex} % para usar bibliografia
\usepackage{csquotes}
\usepackage[portuguese]{babel} %Usar língua portuguesa
\usepackage{blindtext} % Gerar texto automaticamente
\usepackage[printonlyused]{acronym}
\usepackage{hyperref} % para autoref
\usepackage{graphicx}
\usepackage{titlesec}

\bibliography{bibliografia}


\begin{document}
%%
% Definições
%
\def\titulo{Projeto 2: Biblioteca de imagens}
\def\data{08/05/2019}
\def\autores{André Patacas, Gil Teixeira, Sofia Vaz, Luis Andrade}
\def\autorescontactos{(93357) andrepatacas@ua.pt, (88194) gilteixeira@ua.pt, (92968) sofiateixeiravaz@ua.pt, (93159) luisnunoferreirap.a@ua.pt}
\def\versao{1}
\def\departamento{DETI}
\def\empresa{LABI}
\def\logotipo{ua.pdf}

%
%%%%%% CAPA %%%%%%
%
\begin{titlepage}

\begin{center}
\centering
%
\vspace*{50mm}
%
{\Huge \titulo}\\ 
%
\vspace{10mm}
%
{\Large \empresa}\\
%
\vspace{10mm}
%
{\large \autores}\\ 
%
\vspace{30mm}
%
\begin{figure}[h]
\centering
\includegraphics{\logotipo}
\end{figure}
%
\vspace{30mm}
\end{center}
%
\begin{flushright}
\versao
\end{flushright}
\end{titlepage}

%%  Página de Título %%
\title{%
{\Huge\textbf{\titulo}}\\
{\Large \departamento\\ \empresa}
}
%
\author{%
    \autores 
}
%
\date{\data}
%
\maketitle


\pagenumbering{roman}

%%%%%% RESUMO %%%%%%
\begin{abstract}
Este relatório pretende descrever como uma biblioteca de imagens pesquisável foi desenvolvida no âmbito da cadeira de Laboratório de Informática. Esta biblioteca de imagens foi feita com base em Python 3, CSS, JavaScript e HTML. Neste relatório incluímos a divisão geral do trabalho, a explicação geral de métodos criados no \textit{backend} e \textit{frontend} e imagens dos resultados. 
\end{abstract}

%%%%%% Agradecimentos %%%%%%
% Segundo glisc deveria aparecer após conclusão...

\tableofcontents
 \listoftables     % descomentar se necessário
 \listoffigures    % descomentar se necessário


%%%%%%%%%%%%%%%%%%%%%%%%%%%%%%%
\clearpage
\pagenumbering{arabic}

%%%%%%%%%%%%%%%%%%%%%%%%%%%%%%%%
\chapter{Introdução}
\label{chap.introducao}

O frontend da aplicação foi feito em HMTL, CSS e JavaScript e o backend foi feito em Python 3. Esta aplicação foi feita no âmbito de Laboratórios de Informática no ano letivo 2018/2019.
A adicionar às especificações básicas pedidas no o guião sobre regras do segundo projeto, construiu-se ainda suporte para pydocs, para haja uma explicação mais detalhada de cada classe e método do nosso projeto.
Este documento está dividido em quatro capítulos.
Depois desta introdução,
no \autoref{chap.metodologia} é apresentada a metodologia seguida, tendo em conta o trabalho faseado de cada um, 
no \autoref{chap.resultados} são apresentados os resultados obtidos,
sendo estes discutidos no \autoref{chap.analise}.
Finalmente, no \autoref{chap.conclusao} são apresentadas
as conclusões do trabalho.

\chapter{Metodologia}
\label{chap.metodologia}
\section{Definição de tarefas}
Esta fase consistiu em, antes de começar sequer a trabalhar, definir quais membros do grupo fariam o quê, o que pode ser consultado na table seguinte: 
\begin{table}[h!]
\begin{center}
\caption{Divisão geral de tarefas}
\begin{tabular}{l|l}
\hline
\multicolumn{1}{|l|}{Aluno} & \multicolumn{1}{l|}{Tarefa Atribuída} \\ \hline
            André Patacas   & backend                               \\ 
            Gil Teixeira      & frontend                               \\
            Sofia Vaz         & relatório                                \\
            Luis Andrade    & testes unitários e funcionais                     
\end{tabular}
\label{tab:table1}
\end{center}
\end{table}

\section{Backend}
\subsection{DbCommunicator.py}
\paragraph{\_\_init\_\_}
Este método inicializa o objeto que será conectado à base de dados.

\paragraph{get\_dims\_and\_color}
Este método devolve os dados da imagem, a cor média e as dimensões. A cor média é calculada via VSL.

\paragraph{request\_caracteristics}
Este método liga-se ao \textit{website} fornecido pelos docentes, devolvendo um array em que cada elemento é um json com:
\begin{itemize}
\item nome,que é o resultado do \textit{hashing} do conteúdo das imagens para evitar imagens replicadas
\item classe
\item \textit{bounding box}, ou seja, a área onde a classe se verifica
\item "confiança" com a qual o servidor conseguiu classificar a imagem
\end{itemize}

\paragraph{add}
Este método adiciona cada imagem à base de dados. Se uma imagem tiver várias classes será  cortada, tendo cada parte  uma classe diferente, sendo também guardada a original. A base de dados terá a seguinte informação para cada objeto:
\begin{itemize}
\item nome
\item altura
\item largura
\item quantidade média de vermelho
\item quantidade média de verde
\item quantidade média de azul
\item \textit{boundig box}
\item confiança, de 0 a 1, de que a imagem foi classificada corretamente
\end{itemize}

\paragraph{remove}
Ao executar este método, uma dada imagem e todas as suas associadas serão removida da base de dados. É de notar que a imagem apenas poderá ser removida da base de dados se esta existir.

\paragraph{get}
Este método devolve uma imagem baseada no id fornecido como parâmetro. 

\paragraph{request}
Devolve uma string com os dados obtidos via GET.

\paragraph{Setup}
Ambos os métodos \_\_clear\_all\_caution\_\_ e  populate são usados para efeitos de \textit{setup} inicial.

\subsection{app.py}
Este programa serve os conteúdos estáticos HTML, JavaScript e CSS e cria uma API transformando os GET e POST requests em chamadas de funções comunicantes com a base de dados. 

\subsection{put\_example}
Este programa envia uma imagem para a base de dados, não tendo qualquer efeito na maneira como o \textit{website} funciona.

\section{Frontend}
Para uma biblioteca de imagens ser viável, a compatibilidade entre dispositivos é necessária. Por isso, na criação desta, foi sempre tida em conta a usabilidade não só em monitores de computador, mas também em dispositivos móveis. 

\subsection{Home/Class List}
Esta página informa o utilizador de todas as classes já detetadas nesta biblioteca, mostrando alguns exemplos. Ao carregar no nome de uma classe, entra-se numa página em que todas as imagens da classe estão visíveis. Ao carregar numa imagem, o utilizador abre-a, vendo a imagem original e ainda as \textit{bounding boxes} de cada classe presente. A página de JS associada é:
\paragraph{class\_list} 
Lista as classes já detetadas e apresenta imagens com exemplos.

\subsection{List Images}
Esta página mostra as imagens cortadas presentes na biblioteca em nenhuma ordem particular. Ao fazer \textit{hover} com o rato, é possível ver qual classe estamos a ver e com quanta confiança foi corretamente categorizada. Ao carregar no botão \textit{more}, abrimos a imagem e temos acesso aos dados acima referidos. Pode ver o código desta página no ficheiro list\_images.html. A página de JS associada é:
\paragraph{image\_inspect}
Quando uma imagem é aberta, esta função mostra ao utilizador as \textit{bounding boxes} de cada classe e informação sobre estas. 

\subsection{SendImage}
Dá a possibilidade ao utilizador de inserir as próprias imagens na biblioteca. Para adicionar estes dados, o utilizador pode arrastar a imagem para a página ou abrir o explorador de ficheiros. Se uma classe for detetada, então a base de dados será atualizada com a informação da nova imagem e esta ficará acessível a todos os utilizadores da biblioteca. A página de JS associada é:
\paragraph{send\_image}
Envia a imagem via POST request para o \textit{backend} do \textit{website}. Quando no backend, será tratada como se fizesse parte dos dados locais.

\subsection{Search Images}
Esta página apresenta o motor de busca da biblioteca. É possível pesquisar por classe (ou seja, por objeto) e por cor. A pesquisa por classe é feita a partir de uma caixa de texto em que são introduzidos dados, enquanto que a pesquisa por coir é feita a partir de um seletor de cor, tornando a experiência o mais intuitiva possível para o utilizador. Para além disso, é possível definir a confiança mínima na deteção da classe para a imagem ser relevante na busca e ainda a quantidade média de cor mínima. A página de JS associada é:
\paragraph{image\_loader}
Mostra as imagens segundo os dados inseridos pelo utilizador. Pesquisará segundo a classe se dados forem inseridos dados na caixa de texto. Se a caixa de \textit{Class Detection Confidence} estiver \textit{ticked}, então apenas serão mostradas imagens com uma confiança igual ou superior à selecionada. Se a caixa \textit{Search with color} estiver ativa, então as imagens surgirão segundo a sua proximidade com a cor selecionada. A caixa \textit{Color Confidence} tem funcionamento parecido com a \textit{Class Detection Confidence}.

\subsection{About}
Aqui é possível encontrar os dados de todos os membros do grupo(nome, número mecanográfico, e-mail institucional) e a percentagem do trabalho feito por cada um.


\chapter{Resultados}
\label{chap.resultados}
Nesta parte do relatório estão colocadas várias capturas de ecrã do \textit{website} e da consola deste.
\begin{figure}[h]
\includegraphics[width=\textwidth]{HomeTOP.png}
\caption{Topo da página Home/Class List}
\label{Fig1}
\end{figure}

\begin{figure}[h]
\includegraphics[width=\textwidth]{HomeBOTTOM.png}
\caption{Página Home/Class List com as imagens e classes visíveis}
\label{Fig2}
\end{figure}

\begin{figure}[h]
\includegraphics[width=\textwidth]{dataImage.png}
\caption{Dados de uma imagem aberta}
\label{Fig3}
\end{figure}

\begin{figure}[h]
\includegraphics[width=\textwidth]{ExemploHOVER.png}
\caption{Efeito \textit{hover} numa imagem}
\label{Fig4}
\end{figure}

\begin{figure}[h]
\includegraphics[width=\textwidth]{ListImages.png}
\caption{Página em que todas as imagens estão listadas}
\label{Fig5}
\end{figure}

\begin{figure}[h]
\includegraphics[width=\textwidth]{SearchDefault.png}
\caption{Página de pesquisa sem dados inseridos}
\label{Fig6}
\end{figure}

\begin{figure}[h]
\includegraphics[width=\textwidth]{SearchWithColor.png}
\caption{Página de pesquisa por cor}
\label{Fig7}
\end{figure}

\begin{figure}[h]
\includegraphics[width=\textwidth]{Send.png}
\caption{Página de submissão de imagens}
\label{Fig8}
\end{figure}

\begin{figure}[h]
\includegraphics[width=\textwidth]{ResultadosInspect.png}
\caption{Página de inspeção de uma imagem com os vários GET requests}
\label{Fig9}
\end{figure}

\chapter{Análise}
\label{chap.analise}
Analisa os resultados.

\chapter{Conclusões}
\label{chap.conclusao}
Apresenta conclusões.

\chapter*{Contribuições dos autores}
Resumir aqui o que cada autor fez no trabalho.
Usar abreviaturas para identificar os autores,
por exemplo AS para António Silva.
No fim indicar a percentagem de contribuição de cada autor.

%%%%%%%%%%%%%%%%%%%%%%%%%%%%%%%%%
\chapter*{Acrónimos}
\begin{acronym}
\acro{ua}[UA]{Universidade de Aveiro}
\acro{miect}[MIECT]{Mestrado Integrado em Engenharia de Computadores e Telemática}
\acro{lei}[LEI]{Licenciatura em Engenharia Informática}
\acro{glisc}[GLISC]{Grey Literature International Steering Committee}
\end{acronym}


%%%%%%%%%%%%%%%%%%%%%%%%%%%%%%%%%
\printbibliography

\end{document}
