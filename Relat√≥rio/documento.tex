\documentclass{report}
\usepackage[T1]{fontenc} % Fontes T1
\usepackage[utf8]{inputenc} % Input UTF8
\usepackage[backend=biber, style=ieee]{biblatex} % para usar bibliografia
\usepackage{csquotes}
\usepackage[portuguese]{babel} %Usar língua portuguesa
\usepackage{blindtext} % Gerar texto automaticamente
\usepackage[printonlyused]{acronym}
\usepackage{hyperref} % para autoref
\usepackage{graphicx}
\usepackage{titlesec}

\bibliography{bibliografia}


\begin{document}
%%
% Definições
%
\def\titulo{Projeto 2: Biblioteca de imagens}
\def\data{08/05/2019}
\def\autores{André Patacas, Gil Teixeira, Sofia Vaz, Luis Andrade}
\def\autorescontactos{(93357) andrepatacas@ua.pt, (88194) gilteixeira@ua.pt, (92968) sofiateixeiravaz@ua.pt, (93159) luisnunoferreirap.a@ua.pt}
\def\versao{1}
\def\departamento{DETI}
\def\empresa{LABI}
\def\logotipo{ua.pdf}

%
%%%%%% CAPA %%%%%%
%
\begin{titlepage}

\begin{center}
\centering
%
\vspace*{50mm}
%
{\Huge \titulo}\\ 
%
\vspace{10mm}
%
{\Large \empresa}\\
%
\vspace{10mm}
%
{\large \autores}\\ 
%
\vspace{30mm}
%
\begin{figure}[h]
\centering
\includegraphics{\logotipo}
\end{figure}
%
\vspace{30mm}
\end{center}
%
\begin{flushright}
\versao
\end{flushright}
\end{titlepage}

%%  Página de Título %%
\title{%
{\Huge\textbf{\titulo}}\\
{\Large \departamento\\ \empresa}
}
%
\author{%
    \autores 
}
%
\date{\data}
%
\maketitle


\pagenumbering{roman}

%%%%%% RESUMO %%%%%%
\begin{abstract}
Este relatório pretende descrever uma biblioteca de imagens pesquisável desenvolvida. 
Nesta, insere-se o tipo de imagem que se procura, que por sua vez é um dos dados da base de dados que tem todas as imagens. Depois, as imagens pedidas pelo utilizador são apresentadas por ordem de relevância.
\end{abstract}

%%%%%% Agradecimentos %%%%%%
% Segundo glisc deveria aparecer após conclusão...

\tableofcontents
 %\listoftables     % descomentar se necessário
% \listoffigures    % descomentar se necessário


%%%%%%%%%%%%%%%%%%%%%%%%%%%%%%%
\clearpage
\pagenumbering{arabic}

%%%%%%%%%%%%%%%%%%%%%%%%%%%%%%%%
\chapter{Introdução}
\label{chap.introducao}

O frontend da aplicação foi feito em HMTL, CSS e JavaScript, enquanto que o backend foi feito em Python 3. Esta aplicação foi feita no âmbito de Laboratórios de Informática no ano letivo 2018/2019.
A adicionar às especificações básicas pedidas, segundo o guião sobre regras do segundo projeto, construiu-se ainda suporte para pydocs, para haja uma explicação mais detalhada de cada classe e método do nosso projeto.
Este documento está dividido em quatro capítulos.
Depois desta introdução,
no \autoref{chap.metodologia} é apresentada a metodologia seguida, tendo em conta o trabalho faseado de cada um, 
no \autoref{chap.resultados} são apresentados os resultados obtidos,
sendo estes discutidos no \autoref{chap.analise}.
Finalmente, no \autoref{chap.conclusao} são apresentadas
as conclusões do trabalho.

\chapter{Metodologia}
\label{chap.metodologia}
\section{Definição de tarefas}
Esta fase consistiu em, antes de começar sequer a trabalhar, definir quais membros do grupo fariam o quê, o que pode ser consultado na table seguinte: 
\begin{table}[h!]
\begin{center}
\caption{Divisão geral de tarefas}
\label{tab:table1}
\begin{tabular}{l|l}
\hline
\multicolumn{1}{|l|}{Aluno} & \multicolumn{1}{l|}{Tarefa Atribuída} \\ \hline
            André Patacas   & backend                               \\ 
            Gil Teixeira      & frontend                               \\
            Sofia Vaz         & relatório                                \\
            Luis Andrade    & testes unitários e funcionais                     
\end{tabular}
\end{center}
\end{table}

\section{Backend}
\subsection{DbCommunicator.py}
\paragraph{\_\_init\_\_}
Este método inicializa o objeto que será conectado à base de dados.

\paragraph{get\_dims\_and\_color}
Este método devolve os dados da imagem, sendo estes a cor média e as dimensões, como o nome indica. 

\paragraph{request\_caracteristics}
Este método liga-se ao \textit{website} fornecido pelos docentes, devolvendo um array em que cada elemento é um json que tem os seguintes atributos:
\begin{itemize}
\item nome,sendo que este passa pelo processo de \textit{hashing} para evitar imagens replicadas
\item classe
\item \textit{bounding box}, ou seja, a área onde a classe se verifica
\item "confiança" com a qual o servidor conseguiu classificar a imagem, isto é, de 0 a 1, sendo 1 o valor máximo, a certeza de que a imagem foi classificada corretamente
\end{itemize}

\paragraph{add}
Este método adiciona cada imagem à base de dados. Se uma imagem tiver várias classes será  cortada, tendo cada parte  uma classe, diferente, sendo também guardada a original. A base de dados terá a seguinte informação para cada objeto:
\begin{itemize}
\item nome
\item altura
\item largura
\item quantidade média de vermelho
\item quantidade média de verde
\item quantidade média de azul
\item \textit{boundig box}
\item confiança, de 0 a 1, da imagem ter sido classificada corretamente
\end{itemize}

\paragraph{remove}
Ao executar este método, uma dada imagem será removida da base de dados e todas as suas imagens associadas. É de notar que a imagem apenas poderá ser removida da base de dados se esta existir.

\paragraph{get}
Este método devolve uma imagem baseado no id fornecido como parâmetro. 

\paragraph{request}
Devolve uma string que pode ou não estar formatada com os dados obtidos via GET.

\paragraph{Setup}
Ambos os métodos \_\_clear\_all\_caution\_\_ e  populate são usados para efeitos de \textit{setup} inicial.

\subsection{app.py}
Tudo o que este programa faz é estabelecer a ligação entre as páginas HMTL, também estabelecendo ligação com o servidor e com a base de dados. 

\subsection{put\_example}
Este programa é um exemplo de como executar o programa, não tendo qualquer efeito na maneira como o \textit{website} funciona.

\chapter{Resultados}
\label{chap.resultados}
Descreve os resultados obtidos.

\chapter{Análise}
\label{chap.analise}
Analisa os resultados.

\chapter{Conclusões}
\label{chap.conclusao}
Apresenta conclusões.

\chapter*{Contribuições dos autores}
Resumir aqui o que cada autor fez no trabalho.
Usar abreviaturas para identificar os autores,
por exemplo AS para António Silva.
No fim indicar a percentagem de contribuição de cada autor.

%%%%%%%%%%%%%%%%%%%%%%%%%%%%%%%%%
\chapter*{Acrónimos}
\begin{acronym}
\acro{ua}[UA]{Universidade de Aveiro}
\acro{miect}[MIECT]{Mestrado Integrado em Engenharia de Computadores e Telemática}
\acro{lei}[LEI]{Licenciatura em Engenharia Informática}
\acro{glisc}[GLISC]{Grey Literature International Steering Committee}
\end{acronym}


%%%%%%%%%%%%%%%%%%%%%%%%%%%%%%%%%
\printbibliography

\end{document}
